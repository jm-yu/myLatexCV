%%%%%%%%%%%%%%%%%%%%%%%%%%%%%%%%%%%%%%%%%
% Medium Length Professional CV
% LaTeX Template
% Version 2.0 (8/5/13)
%
% This template has been downloaded from:
% http://www.LaTeXTemplates.com
%
% Original author:
% Trey Hunner (http://www.treyhunner.com/)
%
% Important note:
% This template requires the resume.cls file to be in the same directory as the
% .tex file. The resume.cls file provides the resume style used for structuring the
% document.
%
%%%%%%%%%%%%%%%%%%%%%%%%%%%%%%%%%%%%%%%%%

%----------------------------------------------------------------------------------------
%	PACKAGES AND OTHER DOCUMENT CONFIGURATIONS
%----------------------------------------------------------------------------------------

\documentclass{resume} % Use the custom resume.cls style
\usepackage[left=.60in,top=.60in,right=0.60in,bottom=0.60in]{geometry} % Document margins



\name{JINGMIN YU} % Your name
\address{jmyu1994@ufl.edu \quad \quad 352-222-1481 
}

\address{3700 SW 27th St D-306, Gainesville, FL, 32608} % Your address
\begin{document}
\vspace{-1.5em}

%----------------------------------------------------------------------------------------
%	EDUCATION SECTION
%----------------------------------------------------------------------------------------

\begin{rSection}{Education}
{\bf M.Sc. Computer Science} \hfill {expected May. 2018}\\
{\bf University of Florida(UF), Gainesville, FL}\\
{\bf GPA 3.6/4.0}{\qquad Major Courses: Analysis of Algorithms, Advanced Data Structure, Cloud Computing, Computer Graphics, Computer Architecture}\vspace{0.4em}\\
{\bf B.Sc. Electronic Engineering  } \hfill { Sep. 2012 - Jun. 2016} \\
{\bf University of Science and Technology of China(USTC), Hefei, China} \\
%/{\bf Department : Electronic Engineering \& Information Science (EEIS)}\\
{\bf GPA 3.2/4.3}{\qquad Major Courses: Embedded Systems and Applications, Fundamentals of Speech Signal Processing, Fundamentals of Signal Statistical Modeling}
\end{rSection}


%\begin{rSection}{Research Interests}
%{\bf Wireless Communication;  Signal processing;  Machine Learning; Pattern Recognition}
%\end{rSection}



\begin{rSection}{Research Experience}

%\begin{rSubsection}{Design of precoders based on max-dmin for MIMO}{Jan. 2015 - Present}{Supervised by Associate Professor Xiaodong Xu}{Dept. of EEIS, USTC}
%\item  Laboratory: Communication and Electronic System Laboratory of USTC.
%\item Learned MIMO linear precoding techniques and extension of max-dmin precoder for
%high-order QAM modulations as well as large MIMO systems.
%\item Trying to come up with a better plan to design precoders based on max-dmin and
%consider the extension from unicast scenario to multicast scenario.
%\end{rSubsection}
\begin{rSubsection}{Clustering on Frequency Hopping Signals}{Jan. 2015 - Jun. 2016}{Supervised by Prof. Xiaodong Xu}{Communication and Electronic System Laboratory of USTC}
\item implemented k-means and a density-based clustering algorithm for analyzing frequency hopping signals
\item built each step of simulation for analyzing frequency hopping signals on MATLAB
\item resulted in the accuracy of network sorting and station separating both over 0.9 
\end{rSubsection}

%\begin{rSubsection}{Wireless Communication}{Jan. 2015 - May. 2015}{}{}
%\item  learned a MIMO linear precoding technique called max-dmin criterion and sphere decoding technique
%\item  tried to generate a general method facilitating sphere decoding in MIMO systems based on finite alphabet and sparse signal, referring to some algorithms in compressive sensing
%\item  refered algorithms on compressive sensing and other convex optimization to solve the method above but failed
%\end{rSubsection}

%------------------------------------------------

%\begin{rSubsection}{Course project of Statistical Signal Analysis }{Mar. 2015 - Jul. 2015}{Supervised by Professor Zhongfu Ye}{Dept. of EEIS, USTC}
%%\item Referred a paper about
%\item Proved that Bayes risk error was a Bregman divergence, which led to the convenience of proving several properties of BER.
%\end{rSubsection}




%\begin{rSubsection}{Course project of Digital Signal Processing }{ Sep. 2014 - Jan. 2015}{Supervised by Professor Guo Wei}{Dept. of EEIS, USTC}
%\item Used Matlab tools to design filters and analyze the performance of the filters.
%\item Investigated quantization error effects in digital filtering systems.
%\end{rSubsection}

\end{rSection}

\begin{rSection}{Programming Experience}

%\begin{rSubsection}{Course project of Fundamentals of Database Systems }{ Feb. 2015 - Jun. 2015}{Supervised by Doctor Pingbo Yuan}{Dept. of EEIS, USTC}
%\item Learned the principles and core methods of Delphi7 to design  database systems.
%\item Designed a library management database, which could realize the basic functions such as data input, delete and update.
%\end{rSubsection}
\begin{rSubsection}{PageRank on Wikipedia Pages(Scala, Spark, GraphX, AWS)}{Sep. 2017 - Oct. 2017}{}{}
\item built an Apache Spark cluster with HDFS on AWS EC2
\item programmed in Scala and computed the ranks of Wikipedia pages based on 31GB Freebase Wikipedia Extraction data
\item compared performance of pure Spark and GraphX
\end{rSubsection}
\begin{rSubsection}{Google Autocomplete(Java, MapReduce, Docker, MAMP)}{July. 2017 - Aug. 2017}{}{}
\item constructed N-Gram Library from Wikipedia data and created Language Model based on statistical probability, saved data to MySQL database
\item designed a webpage with HTML/CSS, JQuery, PHP and Ajax to present results
\end{rSubsection}
\begin{rSubsection}{Android TodoList(Java, XML, Android Studio)}{June. 2017 - July. 2017}{}{}
\item designed main page with ListView and Fab
\item optimized memory consumption and glide fluency with recycled view and convertView
\item used Android AlarmManager and NotificationManager to remind users of unfinished business
\end{rSubsection}




%\begin{rSubsection}{Course Project for Digital Signal Processing}{Oct. 2014 - Dec. 2015}{}{}
%\item analysed effect of sampling rate and quantization for a speech signal using Matlab and  Adobe Audition
%\item designed a filter with Matlab and dissected quantization effects for an IIR filter
%\end{rSubsection}

\begin{rSubsection}{Internet Chat Application(Java, Eclipse, Swing)}{Oct. 2016 - Nov. 2016}{}{}
\item programmed in Java and utilized Socket Programming on Eclipse
\item realized unicasting and broadcasting message and file on multi-threads
\item designed a concise GUI with Java Swing to manipulate the application and show results
\end{rSubsection}

%\begin{rSubsection}{MIPS Simulator(Java}{Oct. 2016 - Nov. 2016}{}{}
%\item developed a disassembler and a cycle-by-cycle MIPS simulator in Java
%\item implemented advanced pipeline using Tomasulo algorithm with out-of-order execution and in-order commit along with a Branch Predictor using Branch Target Buffer
%\end{rSubsection}


\begin{rSubsection}{Arm-like device(C++, OpenGL, Blender, Xcode)}{Sep. 2016 - Nov. 2016}{}{}
\item used Blender to draw an arm-like object and applied OpenGL to manipulate and show on Xcode
\item accomplished picking and rotating for each part of the device 
\item displayed a 3D model and automate fitting the texture via ray casting
\end{rSubsection}


%\begin{rSubsection}{Library Database GUI}{Apr. 2015 - May. 2015}{}{}
%\item designed a library management database
%\item developed a GUI program with Delphi7 and manipulated Windows Access Database with embedded SQL language

%\item obtained the patience as well as concentration to make a qualified program
%\end{rSubsection}


%----------------------------------------------------------------------------------------
%	EXAMPLE SECTION
%----------------------------------------------------------------------------------------


\end{rSection}

\begin{pSection}{Skills}
\item Languages: Java, Python, C/C++, MATLAB, SQL, Scala, HTML/CSS, Javascript, PHP
\item Frameworks and Tools: Apache Hadoop/Spark, Swing(Java), Maven, jQuery, TensorFlow, Docker, Android Studio, IntelliJ, Git, Linux.
%\item Softwares$\&$Platforms: Matlab, QuartusII, Delphi7, Eclipse, Xcode, Visual Studio
\end{pSection}



%\begin{rSection}{English Proficiency }
%TOFEL iBT: 92      (R: , L: 25, S: 18, W: 23)\\
%GRE : 322 + 3.0 (V152, Q170, AW:3.0)
%\end{rSection}

%----------------------------------------------------------------------------------------

\end{document}
